\section{Einleitung}\label{einleitung}

\begin{enumerate}
  \item Einleitung
  \begin{itemize}
    \item Problemstellung und Zielsetzung
  \end{itemize}
  
  \item Hintergrund
  \\
    - Arnika
    \begin{itemize}
      \item medizinische Nutzung
      \item ökologische Besonderheiten
      \item nachhaltige Ernte
    \end{itemize}
    - Projekt Apuseni
    \begin{itemize}
      \item Projektpartner
      \item Ökologischer Hintergrund
      \item Vorgehensweise
      \item Drohneneinsatz zur Fernerkundung
    \end{itemize}
    - Testflüge im Schwarzwald (Methoden?)
    \begin{itemize}
        \item Welche Art von Drohne (Schnelligkeit/Höhe vs Bildqualität)
        \item Warum Monitoring mit Hilfe von Drohnen (Vorteile etc)
        \item Durchführung (Flugplanung und Beispielbilder beschreiben)
        \item entstandene Bilder beurteilen, Erfolgreich?
        
    \end{itemize}
    
  \item Monitoringsoftware
    
      - Programmiersprache %(Verweis auf Einstiegsmöglichkeit in die verwendete Programmiersprache-> Anhang)
      
      - Anwendungsbereich

      - Methoden %Vorgehensweise
        \begin{itemize}
        \item Parameterbeschreibung
        \end{itemize}
      - Programmaufbau %(Flow-chart und Beschreibung der einzelnen Bestandteile und der Verknüpfungen zwischen den Modulen)

      - Implementierung / Aufruf
        \begin{itemize}
          \item Hard- und Softwarevoraussetzungen
          \item In- und Output-Dateien %und deren Formate      
        \end{itemize}
      %\item Einbindungsmöglichkeiten in andere Anwendungen
      %\item Bei komplexen Berechnungsverfahren Beschreibung der Algorithmen und Literaturhinweise zu den Quellen der Verfahren

    
    
  \item Ergebnisse
    \begin{itemize}
      \item Auswertung
      \item Alle Ergebnisse werden hinsichtlich der Zielsetzung (Anfangshypothesen)
bewertet
      \item Alle Ergebnisse werden mit relevanter Literatur verglichen und bewertet
      \item Verbesserungsmöglichkeiten
      \item Schwierigkeiten Probleme (-> Verwechslungspotential Arnika?) 
      \item Die eigenen Ergebnisse werden nicht über- oder unterbewertet
    \end{itemize}
  \item Diskussion?
  \item Schluss
    \begin{itemize}
      \item Zusammenfassung
      \item Fazit Schlussfolgerungen
      \item Rückbezug auf Ziele
      \item Evaluation
      \item Ausblick
    \end{itemize}
  

\end{enumerate}

\subsection{Ziele}\label{ziele}

Das Ziel der Arbeit ist die Entwicklung einer Anwendungssoftware, die das Monitoring von Arnika Pflanzen erleichtert. Das Programm wird zur Analyse von Luftaufnahmen, die durch Drohnen gemacht wurden, verwendet. Dabei sollen visuelle Erkenntnisse über die Quantität von Arnika Blüten in den Fotos gewonnen werden.
Die Software soll auf Daten mit unterschiedlicher Qualität und Größe anwendet werden können. 


%ich muss konkretisieren dass meine "Anwendungssoftware" nicht gleich die Software des Projektes ist. also kurz meine Relation zu dem Projekt sagen und eine kritische EInschätzung was mein Programm kann oder was nicht.
%Und zwar schon in der Einleitung. Und dann dass ich dem projekt zuarbeite, bzw. 
