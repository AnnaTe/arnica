
\lstset{language=java}
\begin{lstlisting}[frame=htrbl, caption={Das Listing zeigt Java Quellcode}, label={lst:result2}]
/* generate TagCloud */
Cloud cloud = new Cloud();
cloud.setMaxWeight(_maxSizeOfText);
cloud.setMinWeight(_minSizeOfText);
cloud.setTagCase(Case.LOWER);
	    
/* evaluate context and find additional stopwords */
String query = getContextQuery(_context);
List<String> contextStoplist = new ArrayList<String>();
contextStoplist = getStopwordsFromDB(query);
	    
/* append context stoplist */
while(contextStoplist != null && !contextStoplist.isEmpty())
  _stoplist.add(contextStoplist.remove(0));
	    
/* add cloud filters */
if (_stoplist != null) {
  DictionaryFilter df = new DictionaryFilter(_stoplist);
  cloud.addInputFilter(df);
}
/* remove empty tags */
NonNullFilter<Tag> nnf = new NonNullFilter<Tag>();
cloud.addInputFilter(nnf);

/* set minimum tag length */
MinLengthFilter mlf = new MinLengthFilter(_minTagLength);
cloud.addInputFilter(mlf);

/* add taglist to tagcloud */
cloud.addText(_taglist);

/* set number of shown tags */	    
cloud.setMaxTagsToDisplay(_tagsToDisplay);
\end{lstlisting}


% Beispiel für Formeln
Die Zuordnung aller möglichen Werte, welche eine Zufallsvariable annehmen kann nennt man \emph{Verteilungsfunktion} von $X$.

\begin{quotation}
Die Funktion F: $\mathbb{R} \rightarrow$ [0,1] mit $F(t) = P (X \le t)$ heißt Verteilungsfunktion von $X$.\footnote{Konen, vgl.}
\end{quotation}

\begin{quotation}
Für eine stetige Zufallsvariable $X: \Omega \rightarrow \mathbb{R}$ heißt eine integrierbare, nichtnegative reelle Funktion $w: \mathbb{R} \rightarrow \mathbb{R}$ mit $F(x) = P(X \le x) = \int_{-\infty}^{x} w(t)dt$ die \emph{Dichte} oder \emph{Wahrscheinlichkeitsdichte} der Zufallsvariablen $X$.\footnote{Konen, vgl.~}
\end{quotation}
